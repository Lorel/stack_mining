\documentclass[11pt,a4paper]{article}
\usepackage[utf8]{inputenc}
\usepackage[T1]{fontenc}

\usepackage[francais,british,UKenglish,USenglish,english,american]{babel}
\usepackage[left=2cm,right=2cm,top=2cm,bottom=2cm]{geometry}
\usepackage{graphicx}
\usepackage{helvet}
\usepackage[hidelinks]{hyperref}
\usepackage{listings}
\usepackage{url}
\usepackage{xcolor}

\renewcommand{\familydefault}{\sfdefault}

\author{Dorian Burihabwa \and Aurelien Havet}
\title{Apply Java stack trace mining to a Stackoverflow dump to find most relevant questions against a stack trace index}
\date{02/12/2014}

\begin{document}
\maketitle
\tableofcontents
\begin{figure}[h]
\begin{lstlisting}
package main

import "fmt"

func main() {
	fmt.Println("Hello, World!")
}
\end{lstlisting}
\end{figure}
\section{Introduction}
Dans l'ingénierie logicielle, les bugs sont légions, et les rapports de crash sont des informations primordiales pour leurs réparations. Ils contiennent en particulier la pile d'appels de fonctions au moment du crash, dite stack trace, qui aide à la compréhension du contexte dans lequel a lieu l'erreur. Il est utile dans ce cas de pouvoir trouver une source d'information au sujet d'un crash ayant une stack trace similaire et proposant une solution pour éviter que le bug à l'origine de ce crash ne se repoduise.

StackOverflow\cite{SO} est un site dédié aux développeurs. Construit sous forme de blog, il permet à chacun de publier son problème afin que d'autres y répondent. S'il s'agit d'un bug logiciel, la stack trace correspondant à celui-ci peut avoir été publiée afin que d'autres développeurs puissent l'analyser et proposer une solution adéquate au problème. Post après post, ce site est devenu une vraie mine d'informations sur les erreurs logicielles rencontrées par la communauté informatique. Pouvoir trouver une question, et à postériori la solution associée, se rapprochant au maximum d'un problème rencontré par un développeur, est pour ce dernier un véritable gain de temps et de productivité.
\newline

Nous nous proposons ici de structurer et de stocker en base de données des stack traces extraites d'un dump des questions postées sur StackOverflow, afin de pouvoir en ressortir les candidates se rapprochant le plus d'une stack trace recherchée. L'idée est de pouvoir accéder de manière efficace et pertinente aux questions semblables au problème du développeur, et donc aux réponses proposées.
\section{Approche}
\subsection{Structuration des données}
\subsection{Parsing}
\bibliography{model}
\bibliographystyle{plain}
\end{document}
